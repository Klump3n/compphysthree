\documentclass[10pt,a4paper]{article}
%]{report}

\usepackage[a4paper, top=2cm, bottom=2.5cm, left=3cm, right=3cm]{geometry}
%\usepackage[
%    vmarginratio=2:2.5, %Verh�ltnis der oben/unten Seitenr�nder zur automatischen Berechnung
%    paper=a4paper,
%    lmargin=3cm, % mittlerer Rand
%    rmargin=3cm, % �u�erer Rand
%    marginparwidth=2.3cm, % Breite des Marginpars
%    includehead, % Kopfzeile in Berechnung einbeziehen
%    includemp % Marginpar in die Berechnung mit einbeziehen
%]{geometry}
%\setlength\marginparwidth{2.3cm} %Die wird sp�ter zum Rechnen gebraucht, wird aber durch die Angabe im geometry package nicht automatisch richtig gesetzt.


%============================================================
% Pakete
%============================================================

\usepackage[english, ngerman]{babel}    % mehrsprachiger Textsatz
\usepackage[utf8]{inputenc}       % Eingabekodierung Parameter latin1 darf ge�ndert werden
\usepackage[T1]{fontenc}                % Schriftenkodierung
\usepackage{graphicx}                       % zum Einbinden von Grafiken
                                                                % zum besseren Aussehen am Bildschirm
%\usepackage{verbatimfiles}	% Ganze Dateien als Verbatim einbinden
% \usepackage{programs}	% Ganze Dateien als Verbatim einbinden
%\usepackage{hyphsubst}		% Silbentrennung
%%\usepackage{xcolor,soul}
%%\usepackage{float}				%Sachen an der richtigen Stelle ausgeben
%\restylefloat{figure}				%Abbildungen an der richtigen Stelle ausgeben
%\restylefloat{table}				%Tabellen an der richtigen Stelle ausgeben
%\usepackage{array}				%f�r Tabellen
%\newcolumntype{C}{>{$}c<{$}} 	%Tabellenspalten C mit mathematischem Inhalt
% \usepackage{subfigure}




% Code-Listings print source code
% Mathe-Pakete
\usepackage{amsfonts}
\usepackage{amsmath}
\usepackage{amsthm}		% Theorem-Umgebung, Beweise
\usepackage{amssymb}
\usepackage{cancel}
\usepackage{mathcomp}


%============================================================
% Titel, Autor, Datum
%============================================================
\title{Übung 10 \\Computational Physics III}
\author{Matthias Plock (552335) \and Paul Ledwon (561764)} %\\Otto Normalverbraucher (271828)}
\date{\today}

%============================================================
% Dokument
%============================================================
\begin{document}

% Titel erstellen
\maketitle
\tableofcontents

\pagenumbering{arabic}
\pagestyle{myheadings}                  % bzw. ist fancyhdr zu benutzten

\subsection{Betrachtungen zum Parameter $R(L)$}

Für die Magnetisierung gilt im Falle großer Systeme für die unterschiedlichen
Phasen

\begin{align*}
<|M|^2>\overset{L\to\infty}{=}
\begin{cases}
\chi\, L^{-d} &  \text{ungeordnet/paramagnetisch/symmetrisch}\\
\text{const}\, L^{-\eta} & \text{Kosterlitz–Thouless} \\
|M_0|^2 & \text{ferromagnetisch/Goldstone}
\end{cases}
\end{align*}

Damit ergibt sich
\begin{align*}
R(L)=\frac{<|M|^2>_{2L}}{<|M|^2>_{L}}=\overset{L\to\infty}{=}
\begin{cases}
 2^{-d}&  \text{ungeordnet/paramagnetisch/symmetrisch}\\
 2^{-\eta}& \text{Kosterlitz–Thouless} \\
1 & \text{ferromagnetisch/Goldstone}
\end{cases}
\end{align*}
Hierbei ist $d$ die Dimension des Systems und $\eta=\sigma(\frac{1}{\kappa})$
bzw. im XY-Modell $\eta=\frac{1}{4 \pi \kappa}$.

\subsection{Betrachtungen zum Fehler von $R(L)$}

Für die gaußsche Fehlerfortpflanzung ist eine Normalverteilung des Fehlers
der unterschiedlichen Größen, sowie die Unkorreliertheit der verschiedenen
Größen notwendig.\\
Im Gegensatz zum Binderparameter, der aus Größen aus derselben Messung
berechnet wird und diese dadurch korreliert sind, wird der Parameter $R(L)$
aus Messwerten zwei verschiedener Messungen berechnet, wodurch keine
Korrelation vorliegt. Durch die Normalverteilung der Fehler ist es darum
möglich, zur Bestimmung des Fehlers von $R(L)$ eine Fehlerfortpflanzung
durchzuführen.






\end{document}
